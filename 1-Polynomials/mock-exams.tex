\documentclass[12pt,a4paper]{article}
\usepackage[utf8]{inputenc}
\usepackage[T1]{fontenc}
\usepackage[english]{babel}
\usepackage[top=2cm,bottom=2cm,left=2.5cm,right=2.5cm]{geometry}
\usepackage{amsmath,amssymb,amsthm}
\usepackage{enumitem}
\usepackage{tcolorbox}
\usepackage{tikz}
\usepackage{pgfplots}
\pgfplotsset{compat=1.18}
\usepackage{fancyhdr}
\usepackage{lastpage}
\usepackage{hyperref}

\tcbuselibrary{breakable,skins}

% Header/Footer
\pagestyle{fancy}
\fancyhf{}
\fancyhead[L]{MathTrackX: Polynomials, Functions and Graphs}
\fancyhead[R]{Mock Exam}
\fancyfoot[C]{Page \thepage\ of \pageref{LastPage}}

% Solution box
\newtcolorbox{solution}{
    colback=green!5!white,
    colframe=green!60!black,
    fonttitle=\bfseries,
    title=Solution,
    breakable,
    before skip=6pt,
    after skip=6pt
}

% Exam header
\newcommand{\examheader}[1]{
\begin{center}
\large\textbf{MOCK EXAM #1}\\[4pt]
\textbf{MathTrackX: Polynomials, Functions and Graphs}\\[4pt]
\normalsize Duration: 60 minutes \quad|\quad Total: 100 points\\[6pt]
\hrule
\vspace{4pt}
\textit{Show all work. Partial credit may be awarded for correct reasoning.}\\
\textit{Covers: Numbers, Linear Functions, Polynomials, Functions \& Graphs}
\end{center}
\vspace{8pt}
}

\title{\textbf{MathTrackX: Mock Examinations}\\[0.5cm]
\large Polynomials, Functions and Graphs\\
Based on University of Adelaide Curriculum}
\author{SOLTANI Achraf}
\date{\today}

\begin{document}

\maketitle
\tableofcontents
\newpage

%=============================================================================
%=============================================================================
\section*{Mock Exam 1}
\addcontentsline{toc}{section}{Mock Exam 1}
%=============================================================================
%=============================================================================

\examheader{1}

\subsection*{Part A: Numbers and Arithmetic (25 points)}

\textbf{Question 1} (10 points)

\begin{enumerate}[label=(\alph*)]
    \item Classify each number as natural, integer, rational, or irrational (most specific): $\sqrt{25}$, $-\frac{3}{7}$, $\pi$, $0.\overline{6}$ \hfill (4 pts)
    \item Simplify: $\frac{2}{3} + \frac{5}{4} - \frac{1}{6}$ \hfill (3 pts)
    \item Evaluate: $4 + 2 \times 3^2 - (8 - 3)$ \hfill (3 pts)
\end{enumerate}

\vspace{2cm}

\textbf{Question 2} (15 points)

\begin{enumerate}[label=(\alph*)]
    \item Solve and express in interval notation: $-3x + 5 \leq 14$ \hfill (5 pts)
    \item Solve: $2(x - 3) > 4x + 2$ \hfill (5 pts)
    \item Evaluate: $|{-4}| + |3 - 8| - |2|$ \hfill (5 pts)
\end{enumerate}

\vspace{2cm}

\subsection*{Part B: Linear Functions (25 points)}

\textbf{Question 3} (10 points)

\begin{enumerate}[label=(\alph*)]
    \item Find the slope of the line through $(-1, 4)$ and $(3, -2)$. \hfill (4 pts)
    \item Write the equation of a line with slope $-2$ passing through $(1, 5)$. \hfill (6 pts)
\end{enumerate}

\vspace{2cm}

\textbf{Question 4} (15 points)

\begin{enumerate}[label=(\alph*)]
    \item Find the equation of the line through $(2, 1)$ and $(4, 7)$. \hfill (6 pts)
    \item Find the equation of the line perpendicular to $y = 3x - 2$ passing through $(6, 1)$. \hfill (5 pts)
    \item Find the $x$-intercept and $y$-intercept of $2x - 5y = 10$. \hfill (4 pts)
\end{enumerate}

\newpage

\subsection*{Part C: Polynomials (25 points)}

\textbf{Question 5} (10 points)

\begin{enumerate}[label=(\alph*)]
    \item For $P(x) = 3x^3 - 2x^2 + x - 5$, state the degree, leading coefficient, and constant term. \hfill (3 pts)
    \item Expand: $(2x - 3)^2$ \hfill (3 pts)
    \item Expand: $(x + 4)(x - 4)$ \hfill (2 pts)
    \item Find: $P(2)$ where $P(x) = x^2 - 3x + 1$ \hfill (2 pts)
\end{enumerate}

\vspace{2cm}

\textbf{Question 6} (15 points)

\begin{enumerate}[label=(\alph*)]
    \item Solve by factoring: $x^2 - 5x - 14 = 0$ \hfill (5 pts)
    \item Solve using the quadratic formula: $2x^2 - 3x - 1 = 0$ \hfill (6 pts)
    \item Use the discriminant to determine the nature of roots of $x^2 + 2x + 5 = 0$ \hfill (4 pts)
\end{enumerate}

\vspace{2cm}

\subsection*{Part D: Functions and Graphs (25 points)}

\textbf{Question 7} (10 points)

\begin{enumerate}[label=(\alph*)]
    \item Find the domain of $f(x) = \frac{x + 1}{x - 4}$ \hfill (3 pts)
    \item Find the domain of $g(x) = \sqrt{2x - 6}$ \hfill (3 pts)
    \item State the domain and range of $h(x) = x^2 - 4$ \hfill (4 pts)
\end{enumerate}

\vspace{2cm}

\textbf{Question 8} (15 points)

Let $f(x) = x^2 + 1$ and $g(x) = 3x - 2$.

\begin{enumerate}[label=(\alph*)]
    \item Find $(f + g)(x)$ \hfill (3 pts)
    \item Find $(f \circ g)(x)$ \hfill (5 pts)
    \item Find $(g \circ f)(2)$ \hfill (4 pts)
    \item Describe the transformation from $y = x^2$ to $y = (x - 3)^2 + 2$ \hfill (3 pts)
\end{enumerate}

\newpage

%-----------------------------------------------------------------------------
\subsection*{Mock Exam 1 --- Solutions}
%-----------------------------------------------------------------------------

\begin{solution}
\textbf{Question 1}

(a) Classifications:
\begin{itemize}
    \item $\sqrt{25} = 5$: \textbf{Natural}
    \item $-\frac{3}{7}$: \textbf{Rational}
    \item $\pi$: \textbf{Irrational}
    \item $0.\overline{6} = \frac{2}{3}$: \textbf{Rational}
\end{itemize}

(b) $\frac{2}{3} + \frac{5}{4} - \frac{1}{6} = \frac{8}{12} + \frac{15}{12} - \frac{2}{12} = \frac{21}{12} = \boxed{\frac{7}{4}}$

(c) $4 + 2 \times 9 - 5 = 4 + 18 - 5 = \boxed{17}$
\end{solution}

\begin{solution}
\textbf{Question 2}

(a) $-3x + 5 \leq 14 \Rightarrow -3x \leq 9 \Rightarrow x \geq -3$

Answer: $\boxed{[-3, \infty)}$

(b) $2x - 6 > 4x + 2 \Rightarrow -2x > 8 \Rightarrow x < -4$

Answer: $\boxed{(-\infty, -4)}$

(c) $|{-4}| + |{-5}| - |2| = 4 + 5 - 2 = \boxed{7}$
\end{solution}

\begin{solution}
\textbf{Question 3}

(a) $m = \frac{-2 - 4}{3 - (-1)} = \frac{-6}{4} = \boxed{-\frac{3}{2}}$

(b) Using point-slope: $y - 5 = -2(x - 1)$

$y = -2x + 2 + 5 = \boxed{y = -2x + 7}$
\end{solution}

\begin{solution}
\textbf{Question 4}

(a) Slope: $m = \frac{7-1}{4-2} = 3$

Point-slope: $y - 1 = 3(x - 2) \Rightarrow \boxed{y = 3x - 5}$

(b) Perpendicular slope: $m_\perp = -\frac{1}{3}$

$y - 1 = -\frac{1}{3}(x - 6) \Rightarrow y = -\frac{1}{3}x + 3 \Rightarrow \boxed{y = -\frac{1}{3}x + 3}$

(c) $x$-intercept: $2x = 10 \Rightarrow x = 5$, point $(5, 0)$

$y$-intercept: $-5y = 10 \Rightarrow y = -2$, point $(0, -2)$
\end{solution}

\begin{solution}
\textbf{Question 5}

(a) Degree: 3, Leading coefficient: 3, Constant term: $-5$

(b) $(2x - 3)^2 = 4x^2 - 12x + 9$

(c) $(x + 4)(x - 4) = x^2 - 16$

(d) $P(2) = 4 - 6 + 1 = \boxed{-1}$
\end{solution}

\begin{solution}
\textbf{Question 6}

(a) $x^2 - 5x - 14 = (x - 7)(x + 2) = 0$

$\boxed{x = 7 \text{ or } x = -2}$

(b) $x = \frac{3 \pm \sqrt{9 + 8}}{4} = \frac{3 \pm \sqrt{17}}{4}$

$\boxed{x = \frac{3 + \sqrt{17}}{4} \text{ or } x = \frac{3 - \sqrt{17}}{4}}$

(c) $\Delta = 4 - 20 = -16 < 0$

\textbf{No real roots} (two complex conjugate roots)
\end{solution}

\begin{solution}
\textbf{Question 7}

(a) Undefined when $x - 4 = 0$: Domain = $\boxed{(-\infty, 4) \cup (4, \infty)}$

(b) Requires $2x - 6 \geq 0 \Rightarrow x \geq 3$: Domain = $\boxed{[3, \infty)}$

(c) Domain: $(-\infty, \infty)$; Range: $[-4, \infty)$ (minimum at vertex $(0, -4)$)
\end{solution}

\begin{solution}
\textbf{Question 8}

(a) $(f + g)(x) = x^2 + 1 + 3x - 2 = \boxed{x^2 + 3x - 1}$

(b) $(f \circ g)(x) = f(3x - 2) = (3x - 2)^2 + 1 = 9x^2 - 12x + 4 + 1 = \boxed{9x^2 - 12x + 5}$

(c) $f(2) = 5$, $g(f(2)) = g(5) = 15 - 2 = \boxed{13}$

(d) Shift \textbf{right 3 units} and \textbf{up 2 units}. Vertex moves to $(3, 2)$.
\end{solution}

\newpage

%=============================================================================
%=============================================================================
\section*{Mock Exam 2}
\addcontentsline{toc}{section}{Mock Exam 2}
%=============================================================================
%=============================================================================

\examheader{2}

\subsection*{Part A: Numbers and Arithmetic (25 points)}

\textbf{Question 1} (12 points)

\begin{enumerate}[label=(\alph*)]
    \item True or False: Every integer is a rational number. Explain. \hfill (3 pts)
    \item Convert $0.\overline{27}$ to a fraction in lowest terms. \hfill (4 pts)
    \item Simplify: $\frac{5}{8} \times \frac{4}{15} \div \frac{1}{3}$ \hfill (5 pts)
\end{enumerate}

\vspace{2cm}

\textbf{Question 2} (13 points)

\begin{enumerate}[label=(\alph*)]
    \item Solve: $5 - 2x \geq 3x + 15$. Give answer in interval notation. \hfill (5 pts)
    \item Express in interval notation: $-3 < x \leq 7$ \hfill (3 pts)
    \item Solve: $|2x - 1| = 5$ \hfill (5 pts)
\end{enumerate}

\vspace{2cm}

\subsection*{Part B: Linear Functions (25 points)}

\textbf{Question 3} (12 points)

\begin{enumerate}[label=(\alph*)]
    \item A line passes through $(0, -3)$ and $(4, 5)$. Find its equation in slope-intercept form. \hfill (6 pts)
    \item Are the lines $y = 2x + 1$ and $y = -\frac{1}{2}x + 3$ parallel, perpendicular, or neither? Justify. \hfill (6 pts)
\end{enumerate}

\vspace{2cm}

\textbf{Question 4} (13 points)

For the function $f(x) = -2x + 6$:
\begin{enumerate}[label=(\alph*)]
    \item Find $f(0)$, $f(3)$, and $f(-2)$ \hfill (4 pts)
    \item Find the $x$-intercept and $y$-intercept \hfill (4 pts)
    \item Sketch the graph \hfill (5 pts)
\end{enumerate}

\newpage

\subsection*{Part C: Polynomials (25 points)}

\textbf{Question 5} (12 points)

\begin{enumerate}[label=(\alph*)]
    \item Multiply: $(x + 2)(x^2 - 3x + 1)$ \hfill (5 pts)
    \item Expand: $(a + b)^2 - (a - b)^2$ and simplify \hfill (4 pts)
    \item Factor: $x^2 - 9x + 20$ \hfill (3 pts)
\end{enumerate}

\vspace{2cm}

\textbf{Question 6} (13 points)

For the quadratic $f(x) = x^2 - 4x - 5$:
\begin{enumerate}[label=(\alph*)]
    \item Find the roots by factoring \hfill (4 pts)
    \item Find the vertex \hfill (4 pts)
    \item State whether the parabola opens up or down \hfill (2 pts)
    \item State the axis of symmetry \hfill (3 pts)
\end{enumerate}

\vspace{2cm}

\subsection*{Part D: Functions and Graphs (25 points)}

\textbf{Question 7} (12 points)

For $P(x) = (x + 1)^2(x - 3)$:
\begin{enumerate}[label=(\alph*)]
    \item State the degree and leading coefficient \hfill (3 pts)
    \item Find all roots and their multiplicities \hfill (3 pts)
    \item Describe the end behavior \hfill (3 pts)
    \item Does the graph cross or bounce at each root? \hfill (3 pts)
\end{enumerate}

\vspace{2cm}

\textbf{Question 8} (13 points)

Let $f(x) = \sqrt{x}$ and $g(x) = x^2 - 4$.
\begin{enumerate}[label=(\alph*)]
    \item Find $(f \circ g)(x)$ and state its domain \hfill (6 pts)
    \item Find $(g \circ f)(x)$ and state its domain \hfill (4 pts)
    \item Is $f(x) = x^3 - 2x$ continuous everywhere? Why? \hfill (3 pts)
\end{enumerate}

\newpage

%-----------------------------------------------------------------------------
\subsection*{Mock Exam 2 --- Solutions}
%-----------------------------------------------------------------------------

\begin{solution}
\textbf{Question 1}

(a) \textbf{True}. Every integer $n$ can be written as $\frac{n}{1}$, which is a rational number.

(b) Let $x = 0.\overline{27}$. Then $100x = 27.\overline{27}$.

$99x = 27 \Rightarrow x = \frac{27}{99} = \boxed{\frac{3}{11}}$

(c) $\frac{5}{8} \times \frac{4}{15} \div \frac{1}{3} = \frac{5}{8} \times \frac{4}{15} \times 3 = \frac{60}{120} = \boxed{\frac{1}{2}}$
\end{solution}

\begin{solution}
\textbf{Question 2}

(a) $5 - 2x \geq 3x + 15 \Rightarrow -5x \geq 10 \Rightarrow x \leq -2$

Answer: $\boxed{(-\infty, -2]}$

(b) $\boxed{(-3, 7]}$

(c) $2x - 1 = 5$ or $2x - 1 = -5$

$2x = 6$ or $2x = -4$

$\boxed{x = 3 \text{ or } x = -2}$
\end{solution}

\begin{solution}
\textbf{Question 3}

(a) Slope: $m = \frac{5 - (-3)}{4 - 0} = \frac{8}{4} = 2$

$y$-intercept is $(0, -3)$, so $b = -3$

$\boxed{y = 2x - 3}$

(b) Slopes are $m_1 = 2$ and $m_2 = -\frac{1}{2}$

$m_1 \cdot m_2 = 2 \times (-\frac{1}{2}) = -1$

The lines are \textbf{perpendicular} (product of slopes is $-1$).
\end{solution}

\begin{solution}
\textbf{Question 4}

(a) $f(0) = 6$, $f(3) = 0$, $f(-2) = 10$

(b) $y$-intercept: $(0, 6)$

$x$-intercept: $-2x + 6 = 0 \Rightarrow x = 3$, point $(3, 0)$

(c) Line with slope $-2$, passing through $(0, 6)$ and $(3, 0)$.
\end{solution}

\begin{solution}
\textbf{Question 5}

(a) $(x + 2)(x^2 - 3x + 1) = x^3 - 3x^2 + x + 2x^2 - 6x + 2 = \boxed{x^3 - x^2 - 5x + 2}$

(b) $(a^2 + 2ab + b^2) - (a^2 - 2ab + b^2) = 4ab = \boxed{4ab}$

(c) $x^2 - 9x + 20 = \boxed{(x - 4)(x - 5)}$
\end{solution}

\begin{solution}
\textbf{Question 6}

(a) $x^2 - 4x - 5 = (x - 5)(x + 1) = 0$

Roots: $\boxed{x = 5 \text{ and } x = -1}$

(b) Vertex $x = -\frac{-4}{2} = 2$

$f(2) = 4 - 8 - 5 = -9$

Vertex: $\boxed{(2, -9)}$

(c) Opens \textbf{up} ($a = 1 > 0$)

(d) Axis of symmetry: $\boxed{x = 2}$
\end{solution}

\begin{solution}
\textbf{Question 7}

(a) Degree: $2 + 1 = 3$; Leading coefficient: $1$ (positive)

(b) Roots: $x = -1$ (multiplicity 2), $x = 3$ (multiplicity 1)

(c) Odd degree, positive leading coefficient:

As $x \to -\infty$, $P(x) \to -\infty$; As $x \to +\infty$, $P(x) \to +\infty$

(d) At $x = -1$: \textbf{bounces} (even multiplicity)

At $x = 3$: \textbf{crosses} (odd multiplicity)
\end{solution}

\begin{solution}
\textbf{Question 8}

(a) $(f \circ g)(x) = f(x^2 - 4) = \sqrt{x^2 - 4}$

Domain: $x^2 - 4 \geq 0 \Rightarrow x^2 \geq 4 \Rightarrow |x| \geq 2$

Domain: $\boxed{(-\infty, -2] \cup [2, \infty)}$

(b) $(g \circ f)(x) = g(\sqrt{x}) = (\sqrt{x})^2 - 4 = x - 4$

Domain of $f$: $x \geq 0$

Domain: $\boxed{[0, \infty)}$

(c) \textbf{Yes}, $f(x) = x^3 - 2x$ is continuous everywhere because it is a polynomial.
\end{solution}

\newpage

%=============================================================================
%=============================================================================
\section*{Mock Exam 3}
\addcontentsline{toc}{section}{Mock Exam 3}
%=============================================================================
%=============================================================================

\examheader{3}

\subsection*{Part A: Numbers and Arithmetic (25 points)}

\textbf{Question 1} (13 points)

\begin{enumerate}[label=(\alph*)]
    \item List all the number types that $-12$ belongs to (from the set $\{\mathbb{N}, \mathbb{Z}, \mathbb{Q}, \mathbb{R}\}$). \hfill (3 pts)
    \item Evaluate: $\frac{1}{2} + \frac{1}{3} + \frac{1}{4}$ \hfill (4 pts)
    \item Evaluate: $(2^3 - 5)^2 + 4 \times 2 - 1$ \hfill (3 pts)
    \item Simplify: $\frac{|-6| - |4|}{|{-2}|}$ \hfill (3 pts)
\end{enumerate}

\vspace{2cm}

\textbf{Question 2} (12 points)

\begin{enumerate}[label=(\alph*)]
    \item Solve: $\frac{x + 3}{2} < 5$ \hfill (4 pts)
    \item Solve: $-4 \leq 2x + 2 < 8$. Express in interval notation. \hfill (5 pts)
    \item Write in interval notation: All real numbers except $x = 3$ \hfill (3 pts)
\end{enumerate}

\vspace{2cm}

\subsection*{Part B: Linear Functions (25 points)}

\textbf{Question 3} (13 points)

\begin{enumerate}[label=(\alph*)]
    \item A horizontal line passes through $(4, -2)$. What is its equation? \hfill (3 pts)
    \item Find the slope of a line parallel to $3x + 6y = 12$. \hfill (5 pts)
    \item The points $(1, k)$ and $(4, 10)$ lie on a line with slope $2$. Find $k$. \hfill (5 pts)
\end{enumerate}

\vspace{2cm}

\textbf{Question 4} (12 points)

A phone plan charges \$20 per month plus \$0.05 per text message.
\begin{enumerate}[label=(\alph*)]
    \item Write a linear function $C(x)$ for the monthly cost if $x$ messages are sent. \hfill (4 pts)
    \item What is the cost for sending 200 messages? \hfill (3 pts)
    \item How many messages can you send for \$35? \hfill (5 pts)
\end{enumerate}

\newpage

\subsection*{Part C: Polynomials (25 points)}

\textbf{Question 5} (12 points)

\begin{enumerate}[label=(\alph*)]
    \item Expand: $(x - 2)^3$ \hfill (5 pts)
    \item Factor completely: $x^3 - 4x$ \hfill (4 pts)
    \item Use the discriminant to determine how many real solutions $3x^2 - x + 2 = 0$ has. \hfill (3 pts)
\end{enumerate}

\vspace{2cm}

\textbf{Question 6} (13 points)

\begin{enumerate}[label=(\alph*)]
    \item Solve: $x^2 + 6x + 9 = 0$ \hfill (4 pts)
    \item Complete the square to write $f(x) = x^2 + 4x + 1$ in vertex form. \hfill (5 pts)
    \item For $f(x) = -(x - 1)^2 + 4$, state the vertex and whether it's a maximum or minimum. \hfill (4 pts)
\end{enumerate}

\vspace{2cm}

\subsection*{Part D: Functions and Graphs (25 points)}

\textbf{Question 7} (12 points)

\begin{enumerate}[label=(\alph*)]
    \item Find the domain of $f(x) = \frac{3}{\sqrt{x - 1}}$ \hfill (4 pts)
    \item Find the range of $g(x) = x^2 + 3$ \hfill (3 pts)
    \item For $h(x) = -2x^4 + 5x^2 - 1$, describe the end behavior. \hfill (5 pts)
\end{enumerate}

\vspace{2cm}

\textbf{Question 8} (13 points)

\begin{enumerate}[label=(\alph*)]
    \item Given $f(x) = 2x - 1$ and $g(x) = x^2$, find $(f \circ g)(x)$ and $(g \circ f)(x)$. \hfill (6 pts)
    \item Starting with $y = |x|$, describe the transformations to get $y = 2|x - 1| - 3$. \hfill (4 pts)
    \item Is $h(x) = \frac{x}{x-2}$ continuous at $x = 2$? Explain. \hfill (3 pts)
\end{enumerate}

\newpage

%-----------------------------------------------------------------------------
\subsection*{Mock Exam 3 --- Solutions}
%-----------------------------------------------------------------------------

\begin{solution}
\textbf{Question 1}

(a) $-12 \in \{\mathbb{Z}, \mathbb{Q}, \mathbb{R}\}$ (Integer, Rational, Real --- not Natural since negative)

(b) $\frac{1}{2} + \frac{1}{3} + \frac{1}{4} = \frac{6}{12} + \frac{4}{12} + \frac{3}{12} = \boxed{\frac{13}{12}}$

(c) $(8 - 5)^2 + 8 - 1 = 9 + 7 = \boxed{16}$

(d) $\frac{6 - 4}{2} = \frac{2}{2} = \boxed{1}$
\end{solution}

\begin{solution}
\textbf{Question 2}

(a) $\frac{x + 3}{2} < 5 \Rightarrow x + 3 < 10 \Rightarrow x < 7$

Answer: $\boxed{(-\infty, 7)}$

(b) $-4 \leq 2x + 2 < 8$

$-6 \leq 2x < 6$

$-3 \leq x < 3$

Answer: $\boxed{[-3, 3)}$

(c) $\boxed{(-\infty, 3) \cup (3, \infty)}$
\end{solution}

\begin{solution}
\textbf{Question 3}

(a) Horizontal line: $\boxed{y = -2}$

(b) Rewrite: $6y = -3x + 12 \Rightarrow y = -\frac{1}{2}x + 2$

Slope: $-\frac{1}{2}$. Parallel line has same slope: $\boxed{m = -\frac{1}{2}}$

(c) $m = \frac{10 - k}{4 - 1} = 2$

$10 - k = 6 \Rightarrow k = \boxed{4}$
\end{solution}

\begin{solution}
\textbf{Question 4}

(a) $\boxed{C(x) = 20 + 0.05x}$

(b) $C(200) = 20 + 10 = \boxed{\$30}$

(c) $35 = 20 + 0.05x \Rightarrow 15 = 0.05x \Rightarrow x = \boxed{300}$ messages
\end{solution}

\begin{solution}
\textbf{Question 5}

(a) $(x - 2)^3 = x^3 - 3(x^2)(2) + 3(x)(4) - 8 = \boxed{x^3 - 6x^2 + 12x - 8}$

(b) $x^3 - 4x = x(x^2 - 4) = \boxed{x(x - 2)(x + 2)}$

(c) $\Delta = 1 - 24 = -23 < 0$

\textbf{No real solutions} (discriminant negative)
\end{solution}

\begin{solution}
\textbf{Question 6}

(a) $x^2 + 6x + 9 = (x + 3)^2 = 0$

$\boxed{x = -3}$ (repeated root)

(b) $f(x) = x^2 + 4x + 1 = (x^2 + 4x + 4) - 3 = \boxed{(x + 2)^2 - 3}$

Vertex: $(-2, -3)$

(c) Vertex: $(1, 4)$

Since $a = -1 < 0$, parabola opens down.

This is a \textbf{maximum} at $y = 4$.
\end{solution}

\begin{solution}
\textbf{Question 7}

(a) Need $x - 1 > 0$ (strict, since in denominator under square root)

Domain: $\boxed{(1, \infty)}$

(b) $g(x) = x^2 + 3 \geq 3$ for all $x$

Range: $\boxed{[3, \infty)}$

(c) Degree 4 (even), leading coefficient $-2$ (negative)

Both ends: $h(x) \to -\infty$ as $x \to \pm\infty$
\end{solution}

\begin{solution}
\textbf{Question 8}

(a) $(f \circ g)(x) = f(x^2) = 2x^2 - 1$

$(g \circ f)(x) = g(2x - 1) = (2x - 1)^2 = 4x^2 - 4x + 1$

(b) Transformations from $y = |x|$ to $y = 2|x - 1| - 3$:
\begin{enumerate}
    \item Shift right 1 unit: $|x - 1|$
    \item Vertical stretch by factor 2: $2|x - 1|$
    \item Shift down 3 units: $2|x - 1| - 3$
\end{enumerate}

(c) \textbf{No}, $h(x) = \frac{x}{x-2}$ is not continuous at $x = 2$ because the function is undefined there (division by zero). This is an \textbf{infinite discontinuity} (vertical asymptote).
\end{solution}

\end{document}
