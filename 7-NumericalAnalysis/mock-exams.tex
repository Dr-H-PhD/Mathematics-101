\documentclass[12pt,a4paper]{article}
\usepackage[utf8]{inputenc}
\usepackage[T1]{fontenc}
\usepackage[english]{babel}
\usepackage[top=2cm,bottom=2cm,left=2.5cm,right=2.5cm]{geometry}
\usepackage{amsmath,amssymb,amsthm}
\usepackage{enumitem}
\usepackage{tcolorbox}
\usepackage{booktabs}
\usepackage{fancyhdr}
\usepackage{lastpage}
\usepackage{hyperref}

\tcbuselibrary{breakable,skins}

% Header/Footer
\pagestyle{fancy}
\fancyhf{}
\fancyhead[L]{Numerical Analysis}
\fancyhead[R]{Mock Exam}
\fancyfoot[C]{Page \thepage\ of \pageref{LastPage}}

% Solution box
\newtcolorbox{solution}{
    colback=green!5!white,
    colframe=green!60!black,
    fonttitle=\bfseries,
    title=Solution,
    breakable,
    before skip=6pt,
    after skip=6pt
}

% Exam header
\newcommand{\examheader}[1]{
\begin{center}
\large\textbf{MOCK EXAM #1}\\[4pt]
\textbf{Numerical Analysis}\\[4pt]
\normalsize Duration: 90 minutes \quad|\quad Total: 100 points\\[6pt]
\hrule
\vspace{4pt}
\textit{Show all work. Partial credit may be awarded for correct reasoning.}\\
\textit{Calculators permitted. Round answers to 4 decimal places unless stated otherwise.}
\end{center}
\vspace{8pt}
}

\title{\textbf{Numerical Analysis: Mock Examinations}\\[0.5cm]
\large Computational Methods for Mathematical Problems\\
Mathematics 101}
\author{SOLTANI Achraf}
\date{\today}

\begin{document}

\maketitle
\tableofcontents
\newpage

%=============================================================================
%=============================================================================
\section*{Mock Exam 1}
\addcontentsline{toc}{section}{Mock Exam 1}
%=============================================================================
%=============================================================================

\examheader{1}

\subsection*{Part A: Errors and Approximations (20 points)}

\textbf{Question 1} (10 points)

\begin{enumerate}[label=(\alph*)]
    \item The exact value of $e$ is $2.71828182...$. If we use $\tilde{e} = 2.718$, find the absolute error and relative error (as a percentage). \hfill (4 pts)
    \item A computation yields $x = 3.14159$ when the true value is $\pi$. How many significant figures are correct? \hfill (3 pts)
    \item Explain why computing $\sqrt{x+1} - \sqrt{x}$ for large $x$ may lead to loss of significance. Write an equivalent formula that avoids this problem. \hfill (3 pts)
\end{enumerate}

\vspace{2cm}

\textbf{Question 2} (10 points)

\begin{enumerate}[label=(\alph*)]
    \item How many bisection iterations are needed to achieve an error less than $10^{-4}$ starting from the interval $[0, 2]$? \hfill (5 pts)
    \item If a method has quadratic convergence with $|e_{n+1}| \leq 0.5|e_n|^2$, and the initial error is $|e_0| = 0.1$, estimate $|e_3|$. \hfill (5 pts)
\end{enumerate}

\vspace{2cm}

\subsection*{Part B: Root Finding Methods (30 points)}

\textbf{Question 3} (15 points)

Consider $f(x) = x^3 - 2x - 5$.

\begin{enumerate}[label=(\alph*)]
    \item Show that there is a root in the interval $[2, 3]$. \hfill (3 pts)
    \item Perform 3 iterations of the bisection method starting from $[2, 3]$. \hfill (6 pts)
    \item Perform 2 iterations of Newton-Raphson starting from $x_0 = 2$. \hfill (6 pts)
\end{enumerate}

\vspace{2cm}

\textbf{Question 4} (15 points)

\begin{enumerate}[label=(\alph*)]
    \item Use the secant method with $x_0 = 1$ and $x_1 = 2$ to find $x_2$ for $f(x) = x^2 - 3$. \hfill (5 pts)
    \item For solving $x = \cos(x)$ by fixed-point iteration, verify that the iteration converges by checking $|g'(x)|$ at the approximate root $r \approx 0.739$. \hfill (5 pts)
    \item Apply 3 iterations of fixed-point iteration to $x = \cos(x)$ starting from $x_0 = 0.5$. \hfill (5 pts)
\end{enumerate}

\newpage

\subsection*{Part C: Interpolation and Numerical Calculus (25 points)}

\textbf{Question 5} (12 points)

Given the data points: $(0, 1)$, $(1, 3)$, $(2, 7)$.

\begin{enumerate}[label=(\alph*)]
    \item Construct the Lagrange interpolating polynomial $P_2(x)$. \hfill (6 pts)
    \item Use your polynomial to estimate $f(1.5)$. \hfill (3 pts)
    \item If the data comes from $f(x) = x^2 + x + 1$, verify your polynomial is correct. \hfill (3 pts)
\end{enumerate}

\vspace{2cm}

\textbf{Question 6} (13 points)

\begin{enumerate}[label=(\alph*)]
    \item Use the central difference formula to approximate $f'(1)$ for $f(x) = \ln(x)$ with $h = 0.1$. Compare with the exact value. \hfill (5 pts)
    \item Apply Simpson's rule with $n = 4$ to approximate $\int_0^2 x^2\, dx$. Compare with the exact value. \hfill (8 pts)
\end{enumerate}

\vspace{2cm}

\subsection*{Part D: Linear Systems and ODEs (25 points)}

\textbf{Question 7} (12 points)

Solve the following system using Gaussian elimination:
\[
\begin{cases}
x + 2y + z = 9\\
2x - y + 3z = 8\\
3x + y - z = 3
\end{cases}
\]

\vspace{2cm}

\textbf{Question 8} (13 points)

Consider the IVP: $y' = y - t$, $y(0) = 2$.

\begin{enumerate}[label=(\alph*)]
    \item Apply Euler's method with $h = 0.2$ to find $y(0.4)$. \hfill (6 pts)
    \item Apply one step of RK4 with $h = 0.2$ to find $y(0.2)$. \hfill (7 pts)
\end{enumerate}

\newpage

%-----------------------------------------------------------------------------
\subsection*{Mock Exam 1 --- Solutions}
%-----------------------------------------------------------------------------

\begin{solution}
\textbf{Question 1}

(a) Absolute error: $|2.71828182 - 2.718| = \boxed{0.00028182}$

Relative error: $\frac{0.00028182}{2.71828182} \approx 0.000104 = \boxed{0.0104\%}$

(b) $\pi = 3.14159265...$, approximation = $3.14159$

Comparing digit by digit: $3.14159$ matches 6 significant figures (the 6th digit rounds correctly).

$\boxed{6 \text{ significant figures}}$

(c) For large $x$, both $\sqrt{x+1}$ and $\sqrt{x}$ are nearly equal, so their difference loses precision (catastrophic cancellation).

Rationalized formula:
\[\sqrt{x+1} - \sqrt{x} = \frac{(\sqrt{x+1} - \sqrt{x})(\sqrt{x+1} + \sqrt{x})}{\sqrt{x+1} + \sqrt{x}} = \boxed{\frac{1}{\sqrt{x+1} + \sqrt{x}}}\]
\end{solution}

\begin{solution}
\textbf{Question 2}

(a) Error after $n$ iterations: $\frac{b-a}{2^{n+1}} < 10^{-4}$

$\frac{2}{2^{n+1}} < 10^{-4} \Rightarrow 2^{n+1} > 2 \times 10^4 \Rightarrow 2^{n+1} > 20000$

$2^{14} = 16384$, $2^{15} = 32768$

So $n + 1 \geq 15$, thus $\boxed{n = 14}$ iterations.

(b) With $|e_{n+1}| \leq 0.5|e_n|^2$:
\begin{align*}
|e_1| &\leq 0.5(0.1)^2 = 0.005\\
|e_2| &\leq 0.5(0.005)^2 = 0.0000125\\
|e_3| &\leq 0.5(0.0000125)^2 \approx \boxed{7.8 \times 10^{-11}}
\end{align*}
\end{solution}

\begin{solution}
\textbf{Question 3}

(a) $f(2) = 8 - 4 - 5 = -1 < 0$

$f(3) = 27 - 6 - 5 = 16 > 0$

Sign change confirms root in $[2, 3]$. \checkmark

(b) Bisection:
\begin{center}
\begin{tabular}{ccccc}
\toprule
$n$ & $a$ & $b$ & $c$ & $f(c)$ \\
\midrule
1 & 2 & 3 & 2.5 & $5.625$ \\
2 & 2 & 2.5 & 2.25 & $1.8906$ \\
3 & 2 & 2.25 & 2.125 & $0.3457$ \\
\bottomrule
\end{tabular}
\end{center}

After 3 iterations: $\boxed{r \approx 2.125}$

(c) Newton-Raphson: $f'(x) = 3x^2 - 2$, $x_{n+1} = x_n - \frac{f(x_n)}{f'(x_n)}$

$x_0 = 2$: $f(2) = -1$, $f'(2) = 10$

$x_1 = 2 - \frac{-1}{10} = \boxed{2.1}$

$x_1 = 2.1$: $f(2.1) = 9.261 - 4.2 - 5 = 0.061$, $f'(2.1) = 13.23 - 2 = 11.23$

$x_2 = 2.1 - \frac{0.061}{11.23} = \boxed{2.0946}$
\end{solution}

\begin{solution}
\textbf{Question 4}

(a) Secant method: $x_2 = x_1 - f(x_1)\frac{x_1 - x_0}{f(x_1) - f(x_0)}$

$f(1) = 1 - 3 = -2$, $f(2) = 4 - 3 = 1$

$x_2 = 2 - (1)\frac{2-1}{1-(-2)} = 2 - \frac{1}{3} = \boxed{1.6667}$

(b) $g(x) = \cos(x)$, $g'(x) = -\sin(x)$

At $r \approx 0.739$: $|g'(0.739)| = |\sin(0.739)| \approx 0.673 < 1$

Since $|g'(r)| < 1$, the iteration \textbf{converges}. \checkmark

(c) Fixed-point iteration $x_{n+1} = \cos(x_n)$:
\begin{align*}
x_0 &= 0.5\\
x_1 &= \cos(0.5) = 0.8776\\
x_2 &= \cos(0.8776) = 0.6390\\
x_3 &= \cos(0.6390) = 0.8027
\end{align*}

$\boxed{x_3 = 0.8027}$ (converging to $\approx 0.739$)
\end{solution}

\begin{solution}
\textbf{Question 5}

(a) Lagrange basis polynomials:
\begin{align*}
L_0(x) &= \frac{(x-1)(x-2)}{(0-1)(0-2)} = \frac{(x-1)(x-2)}{2}\\
L_1(x) &= \frac{(x-0)(x-2)}{(1-0)(1-2)} = -x(x-2) = -x^2 + 2x\\
L_2(x) &= \frac{(x-0)(x-1)}{(2-0)(2-1)} = \frac{x(x-1)}{2}
\end{align*}

$P_2(x) = 1 \cdot L_0 + 3 \cdot L_1 + 7 \cdot L_2$

$= \frac{(x-1)(x-2)}{2} + 3(-x^2+2x) + 7\frac{x(x-1)}{2}$

$= \frac{x^2-3x+2}{2} - 3x^2 + 6x + \frac{7x^2-7x}{2}$

$= \frac{x^2-3x+2 + 7x^2-7x}{2} - 3x^2 + 6x = \frac{8x^2-10x+2}{2} - 3x^2 + 6x$

$= 4x^2 - 5x + 1 - 3x^2 + 6x = \boxed{x^2 + x + 1}$

(b) $P_2(1.5) = 2.25 + 1.5 + 1 = \boxed{4.75}$

(c) $f(x) = x^2 + x + 1$: $f(0) = 1$, $f(1) = 3$, $f(2) = 7$ \checkmark

The polynomial matches exactly.
\end{solution}

\begin{solution}
\textbf{Question 6}

(a) Central difference: $f'(x) \approx \frac{f(x+h) - f(x-h)}{2h}$

$f'(1) \approx \frac{\ln(1.1) - \ln(0.9)}{0.2} = \frac{0.0953 - (-0.1054)}{0.2} = \frac{0.2007}{0.2} = \boxed{1.0034}$

Exact: $f'(x) = \frac{1}{x}$, so $f'(1) = 1$

Error: $|1.0034 - 1| = 0.0034$

(b) Simpson's rule with $n = 4$, $h = 0.5$:

$x_i$: 0, 0.5, 1, 1.5, 2

$f(x_i) = x_i^2$: 0, 0.25, 1, 2.25, 4

$I \approx \frac{h}{3}[f_0 + 4(f_1 + f_3) + 2f_2 + f_4]$

$= \frac{0.5}{3}[0 + 4(0.25 + 2.25) + 2(1) + 4]$

$= \frac{0.5}{3}[0 + 10 + 2 + 4] = \frac{0.5 \times 16}{3} = \boxed{\frac{8}{3} = 2.6667}$

Exact: $\int_0^2 x^2\, dx = \frac{x^3}{3}\Big|_0^2 = \frac{8}{3}$

Simpson's rule gives the exact answer for polynomials of degree $\leq 3$.
\end{solution}

\begin{solution}
\textbf{Question 7}

Augmented matrix:
$\left[\begin{array}{ccc|c} 1 & 2 & 1 & 9\\ 2 & -1 & 3 & 8\\ 3 & 1 & -1 & 3 \end{array}\right]$

$R_2 \leftarrow R_2 - 2R_1$, $R_3 \leftarrow R_3 - 3R_1$:

$\left[\begin{array}{ccc|c} 1 & 2 & 1 & 9\\ 0 & -5 & 1 & -10\\ 0 & -5 & -4 & -24 \end{array}\right]$

$R_3 \leftarrow R_3 - R_2$:

$\left[\begin{array}{ccc|c} 1 & 2 & 1 & 9\\ 0 & -5 & 1 & -10\\ 0 & 0 & -5 & -14 \end{array}\right]$

Back substitution:
$-5z = -14 \Rightarrow z = \frac{14}{5} = 2.8$

$-5y + 2.8 = -10 \Rightarrow y = 2.56$

$x + 2(2.56) + 2.8 = 9 \Rightarrow x = 1.08$

$\boxed{x = 1.08, \quad y = 2.56, \quad z = 2.8}$
\end{solution}

\begin{solution}
\textbf{Question 8}

(a) Euler's method: $y_{n+1} = y_n + h \cdot f(t_n, y_n) = y_n + 0.2(y_n - t_n)$

$y_0 = 2$, $t_0 = 0$:
$y_1 = 2 + 0.2(2 - 0) = 2 + 0.4 = 2.4$

$y_1 = 2.4$, $t_1 = 0.2$:
$y_2 = 2.4 + 0.2(2.4 - 0.2) = 2.4 + 0.44 = 2.84$

$\boxed{y(0.4) \approx 2.84}$

(b) RK4 with $h = 0.2$, $f(t,y) = y - t$, $t_0 = 0$, $y_0 = 2$:

$k_1 = 0.2 f(0, 2) = 0.2(2) = 0.4$

$k_2 = 0.2 f(0.1, 2.2) = 0.2(2.2 - 0.1) = 0.2(2.1) = 0.42$

$k_3 = 0.2 f(0.1, 2.21) = 0.2(2.21 - 0.1) = 0.2(2.11) = 0.422$

$k_4 = 0.2 f(0.2, 2.422) = 0.2(2.422 - 0.2) = 0.2(2.222) = 0.4444$

$y_1 = 2 + \frac{1}{6}(0.4 + 2(0.42) + 2(0.422) + 0.4444)$

$= 2 + \frac{1}{6}(0.4 + 0.84 + 0.844 + 0.4444) = 2 + \frac{2.5284}{6}$

$\boxed{y(0.2) \approx 2.4214}$
\end{solution}

\newpage

%=============================================================================
%=============================================================================
\section*{Mock Exam 2}
\addcontentsline{toc}{section}{Mock Exam 2}
%=============================================================================
%=============================================================================

\examheader{2}

\subsection*{Part A: Errors and Root Finding (25 points)}

\textbf{Question 1} (10 points)

\begin{enumerate}[label=(\alph*)]
    \item Given that $\sin(0.1) = 0.0998334...$, if we approximate it as $0.1$ (first term of Taylor series), find the absolute and relative errors. \hfill (4 pts)
    \item The quadratic formula involves computing $\sqrt{b^2 - 4ac}$. Explain when catastrophic cancellation might occur in calculating $\frac{-b + \sqrt{b^2-4ac}}{2a}$. \hfill (3 pts)
    \item What is the order of convergence for the bisection method? For Newton-Raphson? \hfill (3 pts)
\end{enumerate}

\vspace{2cm}

\textbf{Question 2} (15 points)

Let $f(x) = e^x - 3x$.

\begin{enumerate}[label=(\alph*)]
    \item Show there is a root in $[0, 1]$. \hfill (3 pts)
    \item Apply 3 iterations of bisection starting from $[0, 1]$. \hfill (6 pts)
    \item Set up the Newton-Raphson iteration formula for this function. \hfill (3 pts)
    \item Starting from $x_0 = 0.5$, perform 2 Newton-Raphson iterations. \hfill (3 pts)
\end{enumerate}

\vspace{2cm}

\subsection*{Part B: Interpolation (25 points)}

\textbf{Question 3} (12 points)

\begin{enumerate}[label=(\alph*)]
    \item Construct the divided difference table for the data:
    \begin{center}
    \begin{tabular}{c|ccc}
    $x$ & 1 & 2 & 4 \\
    \hline
    $f(x)$ & 1 & 3 & 9
    \end{tabular}
    \end{center} \hfill (6 pts)
    \item Write the Newton interpolating polynomial. \hfill (4 pts)
    \item Estimate $f(3)$. \hfill (2 pts)
\end{enumerate}

\vspace{2cm}

\textbf{Question 4} (13 points)

\begin{enumerate}[label=(\alph*)]
    \item State the formula for the interpolation error when using $n+1$ points. \hfill (3 pts)
    \item What is Runge's phenomenon? How can it be mitigated? \hfill (4 pts)
    \item Find the Lagrange interpolating polynomial through $(-1, 4)$, $(0, 1)$, $(1, 0)$. \hfill (6 pts)
\end{enumerate}

\newpage

\subsection*{Part C: Numerical Differentiation and Integration (25 points)}

\textbf{Question 5} (12 points)

\begin{enumerate}[label=(\alph*)]
    \item Write the forward, backward, and central difference formulas for $f'(x)$. Which is most accurate? \hfill (4 pts)
    \item Use the second derivative central difference formula to approximate $f''(2)$ for $f(x) = x^3$ with $h = 0.1$. \hfill (4 pts)
    \item Explain the trade-off between truncation error and round-off error when choosing $h$ for numerical differentiation. \hfill (4 pts)
\end{enumerate}

\vspace{2cm}

\textbf{Question 6} (13 points)

\begin{enumerate}[label=(\alph*)]
    \item Use the trapezoidal rule with $n = 4$ to approximate $\int_1^3 \frac{1}{x}\, dx$. \hfill (6 pts)
    \item What is the exact value of this integral? Calculate the error. \hfill (3 pts)
    \item What is the error order of the trapezoidal rule? Of Simpson's rule? \hfill (4 pts)
\end{enumerate}

\vspace{2cm}

\subsection*{Part D: Linear Systems and ODEs (25 points)}

\textbf{Question 7} (12 points)

\begin{enumerate}[label=(\alph*)]
    \item Apply 2 iterations of the Jacobi method to:
    \[
    \begin{cases}
    5x + y = 11\\
    x + 4y = 18
    \end{cases}
    \]
    starting from $(x_0, y_0) = (0, 0)$. \hfill (6 pts)
    \item Is this system diagonally dominant? Will the iteration converge? \hfill (3 pts)
    \item What is the difference between Jacobi and Gauss-Seidel methods? \hfill (3 pts)
\end{enumerate}

\vspace{2cm}

\textbf{Question 8} (13 points)

Consider $y' = -2y$, $y(0) = 1$.

\begin{enumerate}[label=(\alph*)]
    \item What is the exact solution? \hfill (3 pts)
    \item Use Euler's method with $h = 0.25$ to approximate $y(0.5)$. \hfill (5 pts)
    \item Compare your Euler approximation with the exact value. \hfill (2 pts)
    \item Why is RK4 generally preferred over Euler's method? \hfill (3 pts)
\end{enumerate}

\newpage

%-----------------------------------------------------------------------------
\subsection*{Mock Exam 2 --- Solutions}
%-----------------------------------------------------------------------------

\begin{solution}
\textbf{Question 1}

(a) Absolute error: $|0.0998334 - 0.1| = \boxed{0.0001666}$

Relative error: $\frac{0.0001666}{0.0998334} \approx 0.00167 = \boxed{0.167\%}$

(b) Catastrophic cancellation occurs when $b > 0$ and $b^2 \gg 4ac$ (discriminant close to $b^2$). Then $\sqrt{b^2-4ac} \approx |b|$, and $-b + \sqrt{b^2-4ac}$ involves subtracting nearly equal numbers.

(c) Bisection: \textbf{Linear} (order 1), gains $\sim 1$ bit per iteration.

Newton-Raphson: \textbf{Quadratic} (order 2), digits double each iteration.
\end{solution}

\begin{solution}
\textbf{Question 2}

(a) $f(0) = 1 - 0 = 1 > 0$

$f(1) = e - 3 \approx 2.718 - 3 = -0.282 < 0$

Sign change $\Rightarrow$ root exists in $[0, 1]$. \checkmark

(b) Bisection:
\begin{center}
\begin{tabular}{ccccc}
\toprule
$n$ & $a$ & $b$ & $c$ & $f(c)$ \\
\midrule
1 & 0 & 1 & 0.5 & $0.149$ \\
2 & 0.5 & 1 & 0.75 & $-0.133$ \\
3 & 0.5 & 0.75 & 0.625 & $-0.006$ \\
\bottomrule
\end{tabular}
\end{center}

Root $\approx \boxed{0.625}$

(c) $f'(x) = e^x - 3$

$\boxed{x_{n+1} = x_n - \frac{e^{x_n} - 3x_n}{e^{x_n} - 3}}$

(d) $x_0 = 0.5$: $f(0.5) = 1.649 - 1.5 = 0.149$, $f'(0.5) = 1.649 - 3 = -1.351$

$x_1 = 0.5 - \frac{0.149}{-1.351} = 0.5 + 0.110 = 0.610$

$x_1 = 0.610$: $f(0.610) = 1.840 - 1.830 = 0.010$, $f'(0.610) = -1.160$

$x_2 = 0.610 - \frac{0.010}{-1.160} = \boxed{0.619}$
\end{solution}

\begin{solution}
\textbf{Question 3}

(a) Divided difference table:
\begin{center}
\begin{tabular}{c|c|c|c}
$x$ & $f[x]$ & $f[\cdot,\cdot]$ & $f[\cdot,\cdot,\cdot]$ \\
\hline
1 & 1 & & \\
  &   & $\frac{3-1}{2-1} = 2$ & \\
2 & 3 &  & $\frac{2-2}{4-1} = 0$ \\
  &   & $\frac{9-3}{4-2} = 3$ & \\
4 & 9 & & \\
\end{tabular}
\end{center}

Wait, let me recalculate: $f[x_1,x_2] = \frac{9-3}{4-2} = 3$, $f[x_0,x_1,x_2] = \frac{3-2}{4-1} = \frac{1}{3}$

(b) $P_2(x) = f[x_0] + f[x_0,x_1](x-1) + f[x_0,x_1,x_2](x-1)(x-2)$

$\boxed{P_2(x) = 1 + 2(x-1) + \frac{1}{3}(x-1)(x-2)}$

(c) $P_2(3) = 1 + 2(2) + \frac{1}{3}(2)(1) = 1 + 4 + \frac{2}{3} = \boxed{5.667}$
\end{solution}

\begin{solution}
\textbf{Question 4}

(a) $\boxed{f(x) - P_n(x) = \frac{f^{(n+1)}(\xi)}{(n+1)!}\prod_{i=0}^{n}(x-x_i)}$

for some $\xi$ in the interval containing $x$ and all $x_i$.

(b) \textbf{Runge's phenomenon}: High-degree polynomial interpolation with equally spaced points can cause large oscillations near the endpoints.

\textbf{Mitigation}: Use Chebyshev nodes (clustered near endpoints), use lower-degree piecewise polynomials (splines), or use fewer points.

(c) Lagrange polynomial through $(-1, 4)$, $(0, 1)$, $(1, 0)$:

$L_0(x) = \frac{(x-0)(x-1)}{(-1-0)(-1-1)} = \frac{x(x-1)}{2}$

$L_1(x) = \frac{(x+1)(x-1)}{(0+1)(0-1)} = -(x^2-1) = 1-x^2$

$L_2(x) = \frac{(x+1)(x-0)}{(1+1)(1-0)} = \frac{x(x+1)}{2}$

$P_2(x) = 4 \cdot \frac{x(x-1)}{2} + 1 \cdot (1-x^2) + 0 \cdot \frac{x(x+1)}{2}$

$= 2x^2 - 2x + 1 - x^2 = \boxed{x^2 - 2x + 1} = (x-1)^2$
\end{solution}

\begin{solution}
\textbf{Question 5}

(a) Forward: $f'(x) \approx \frac{f(x+h)-f(x)}{h}$, error $O(h)$

Backward: $f'(x) \approx \frac{f(x)-f(x-h)}{h}$, error $O(h)$

Central: $f'(x) \approx \frac{f(x+h)-f(x-h)}{2h}$, error $O(h^2)$ $\leftarrow$ \textbf{most accurate}

(b) $f''(x) \approx \frac{f(x+h) - 2f(x) + f(x-h)}{h^2}$

$f(1.9) = 6.859$, $f(2) = 8$, $f(2.1) = 9.261$

$f''(2) \approx \frac{9.261 - 16 + 6.859}{0.01} = \frac{0.12}{0.01} = \boxed{12}$

Exact: $f''(x) = 6x$, so $f''(2) = 12$ \checkmark

(c) \textbf{Truncation error} decreases as $h \to 0$ (from Taylor series approximation).

\textbf{Round-off error} increases as $h \to 0$ (subtracting nearly equal numbers).

There is an optimal $h$ that balances these two sources of error.
\end{solution}

\begin{solution}
\textbf{Question 6}

(a) Trapezoidal rule: $h = \frac{3-1}{4} = 0.5$

$x$: 1, 1.5, 2, 2.5, 3

$f(x) = 1/x$: 1, 0.6667, 0.5, 0.4, 0.3333

$I \approx \frac{0.5}{2}[1 + 2(0.6667 + 0.5 + 0.4) + 0.3333]$

$= 0.25[1 + 3.1334 + 0.3333] = 0.25 \times 4.4667 = \boxed{1.1167}$

(b) Exact: $\int_1^3 \frac{1}{x}\, dx = \ln(3) - \ln(1) = \ln(3) \approx 1.0986$

Error: $|1.1167 - 1.0986| = \boxed{0.0181}$

(c) Trapezoidal rule: $\boxed{O(h^2)}$

Simpson's rule: $\boxed{O(h^4)}$
\end{solution}

\begin{solution}
\textbf{Question 7}

(a) Rearranged: $x = \frac{11-y}{5}$, $y = \frac{18-x}{4}$

Jacobi iteration:
\begin{align*}
k=1: \quad x^{(1)} &= \frac{11-0}{5} = 2.2, \quad y^{(1)} = \frac{18-0}{4} = 4.5\\
k=2: \quad x^{(2)} &= \frac{11-4.5}{5} = 1.3, \quad y^{(2)} = \frac{18-2.2}{4} = 3.95
\end{align*}

After 2 iterations: $\boxed{x \approx 1.3, \quad y \approx 3.95}$

(b) Diagonal dominance: $|5| > |1|$ and $|4| > |1|$ \checkmark

Yes, \textbf{diagonally dominant}, so iteration will \textbf{converge}.

(c) \textbf{Jacobi}: Uses values from previous iteration only.

\textbf{Gauss-Seidel}: Uses updated values immediately (faster convergence).
\end{solution}

\begin{solution}
\textbf{Question 8}

(a) $\frac{dy}{y} = -2\, dt \Rightarrow \ln|y| = -2t + C$

$y(0) = 1 \Rightarrow C = 0$

$\boxed{y = e^{-2t}}$

(b) Euler: $y_{n+1} = y_n + h(-2y_n) = y_n(1 - 2h) = 0.5 y_n$

$y_0 = 1$

$y_1 = 0.5(1) = 0.5$

$y_2 = 0.5(0.5) = 0.25$

$\boxed{y(0.5) \approx 0.25}$

(c) Exact: $y(0.5) = e^{-1} \approx 0.3679$

Euler gives $0.25$, error $\approx 0.118$ (32\% relative error)

(d) RK4 is preferred because:
\begin{itemize}
    \item Fourth-order accuracy ($O(h^4)$) vs first-order ($O(h)$)
    \item Much more accurate for same step size
    \item Better stability properties
    \item Cost of 4 function evaluations is usually worthwhile
\end{itemize}
\end{solution}

\newpage

%=============================================================================
%=============================================================================
\section*{Mock Exam 3}
\addcontentsline{toc}{section}{Mock Exam 3}
%=============================================================================
%=============================================================================

\examheader{3}

\subsection*{Part A: Comprehensive Root Finding (25 points)}

\textbf{Question 1} (12 points)

The equation $x = 2\sin(x)$ has a nonzero root near $x = 1.9$.

\begin{enumerate}[label=(\alph*)]
    \item Rewrite as $f(x) = 0$ and verify there is a root in $[1.5, 2]$. \hfill (3 pts)
    \item Perform 2 iterations of Newton-Raphson starting from $x_0 = 1.9$. \hfill (6 pts)
    \item Rewrite as $x = g(x)$ for fixed-point iteration. Will $g(x) = 2\sin(x)$ converge? \hfill (3 pts)
\end{enumerate}

\vspace{2cm}

\textbf{Question 2} (13 points)

\begin{enumerate}[label=(\alph*)]
    \item Compare bisection, Newton-Raphson, and secant methods in terms of:
    \begin{itemize}
        \item Convergence rate
        \item Required information (function values, derivatives)
        \item Reliability
    \end{itemize} \hfill (6 pts)
    \item When would you choose bisection over Newton-Raphson? \hfill (3 pts)
    \item What happens to Newton-Raphson near a root of multiplicity 2? \hfill (4 pts)
\end{enumerate}

\vspace{2cm}

\subsection*{Part B: Interpolation and Approximation (25 points)}

\textbf{Question 3} (13 points)

The following data represents temperature readings:
\begin{center}
\begin{tabular}{c|cccc}
Time (hr) & 0 & 2 & 4 & 6 \\
\hline
Temp (°C) & 20 & 26 & 30 & 28
\end{tabular}
\end{center}

\begin{enumerate}[label=(\alph*)]
    \item Use Lagrange interpolation to estimate the temperature at $t = 3$. \hfill (8 pts)
    \item Would you trust this interpolation for $t = 10$? Explain. \hfill (2 pts)
    \item What degree is the interpolating polynomial? \hfill (3 pts)
\end{enumerate}

\vspace{2cm}

\textbf{Question 4} (12 points)

\begin{enumerate}[label=(\alph*)]
    \item Construct the Newton divided difference polynomial for $(0, 1)$, $(1, 0)$, $(2, 1)$. \hfill (6 pts)
    \item Add the point $(3, 4)$ to your polynomial. What is the advantage of Newton's form for this? \hfill (6 pts)
\end{enumerate}

\newpage

\subsection*{Part C: Numerical Integration (25 points)}

\textbf{Question 5} (13 points)

Approximate $\int_0^{\pi} \sin(x)\, dx$ using:

\begin{enumerate}[label=(\alph*)]
    \item The trapezoidal rule with $n = 2$. \hfill (4 pts)
    \item Simpson's rule with $n = 2$. \hfill (4 pts)
    \item Compare both with the exact value and discuss which is more accurate. \hfill (5 pts)
\end{enumerate}

\vspace{2cm}

\textbf{Question 6} (12 points)

\begin{enumerate}[label=(\alph*)]
    \item Derive the trapezoidal rule by integrating the linear interpolant through $(a, f(a))$ and $(b, f(b))$. \hfill (6 pts)
    \item Simpson's rule is exact for polynomials up to what degree? \hfill (2 pts)
    \item How does the composite trapezoidal rule differ from the simple trapezoidal rule? \hfill (4 pts)
\end{enumerate}

\vspace{2cm}

\subsection*{Part D: Systems and ODEs (25 points)}

\textbf{Question 7} (12 points)

\begin{enumerate}[label=(\alph*)]
    \item Factor $A = LU$ for $A = \begin{pmatrix} 2 & 1 \\ 4 & 5 \end{pmatrix}$. \hfill (6 pts)
    \item Use your LU factorization to solve $A\mathbf{x} = \begin{pmatrix} 5 \\ 13 \end{pmatrix}$. \hfill (6 pts)
\end{enumerate}

\vspace{2cm}

\textbf{Question 8} (13 points)

For the IVP $y' = t + y$, $y(0) = 1$:

\begin{enumerate}[label=(\alph*)]
    \item Compute one step of RK2 (midpoint method) with $h = 0.2$. \hfill (6 pts)
    \item Compute one step of RK4 with $h = 0.2$. \hfill (7 pts)
\end{enumerate}

\newpage

%-----------------------------------------------------------------------------
\subsection*{Mock Exam 3 --- Solutions}
%-----------------------------------------------------------------------------

\begin{solution}
\textbf{Question 1}

(a) $f(x) = x - 2\sin(x) = 0$

$f(1.5) = 1.5 - 2\sin(1.5) = 1.5 - 1.995 = -0.495 < 0$

$f(2) = 2 - 2\sin(2) = 2 - 1.819 = 0.181 > 0$

Sign change $\Rightarrow$ root in $[1.5, 2]$. \checkmark

(b) $f'(x) = 1 - 2\cos(x)$

$x_0 = 1.9$: $f(1.9) = 1.9 - 2\sin(1.9) = 1.9 - 1.892 = 0.008$

$f'(1.9) = 1 - 2\cos(1.9) = 1 - 2(-0.323) = 1.646$

$x_1 = 1.9 - \frac{0.008}{1.646} = 1.895$

$x_1 = 1.895$: $f(1.895) \approx 0.0004$, $f'(1.895) \approx 1.64$

$x_2 = 1.895 - \frac{0.0004}{1.64} = \boxed{1.8955}$

(c) $g(x) = 2\sin(x)$, $g'(x) = 2\cos(x)$

At $r \approx 1.9$: $|g'(1.9)| = |2\cos(1.9)| = 2(0.323) = 0.646 < 1$

Yes, it will \textbf{converge} since $|g'(r)| < 1$.
\end{solution}

\begin{solution}
\textbf{Question 2}

(a) Comparison:
\begin{center}
\begin{tabular}{l|ccc}
& Bisection & Newton & Secant \\
\hline
Convergence & Linear & Quadratic & $\approx 1.618$ \\
Requirements & $f$ only & $f$ and $f'$ & $f$ only \\
Reliability & Always* & May diverge & May diverge \\
\end{tabular}
\end{center}
*if sign change exists

(b) Choose bisection when:
\begin{itemize}
    \item Derivative is expensive or unavailable
    \item Newton-Raphson diverges (bad initial guess)
    \item Robustness is more important than speed
    \item Need guaranteed convergence
\end{itemize}

(c) At a root of multiplicity 2, Newton-Raphson convergence degrades from quadratic to \textbf{linear}. The formula can be modified to $x_{n+1} = x_n - 2\frac{f(x_n)}{f'(x_n)}$ to restore quadratic convergence.
\end{solution}

\begin{solution}
\textbf{Question 3}

(a) Points: $(0, 20)$, $(2, 26)$, $(4, 30)$, $(6, 28)$

At $t = 3$:

$L_0(3) = \frac{(3-2)(3-4)(3-6)}{(0-2)(0-4)(0-6)} = \frac{(1)(-1)(-3)}{(-2)(-4)(-6)} = \frac{3}{-48} = -0.0625$

$L_1(3) = \frac{(3-0)(3-4)(3-6)}{(2-0)(2-4)(2-6)} = \frac{(3)(-1)(-3)}{(2)(-2)(-4)} = \frac{9}{16} = 0.5625$

$L_2(3) = \frac{(3-0)(3-2)(3-6)}{(4-0)(4-2)(4-6)} = \frac{(3)(1)(-3)}{(4)(2)(-2)} = \frac{-9}{-16} = 0.5625$

$L_3(3) = \frac{(3-0)(3-2)(3-4)}{(6-0)(6-2)(6-4)} = \frac{(3)(1)(-1)}{(6)(4)(2)} = \frac{-3}{48} = -0.0625$

$T(3) = 20(-0.0625) + 26(0.5625) + 30(0.5625) + 28(-0.0625)$

$= -1.25 + 14.625 + 16.875 - 1.75 = \boxed{28.5°C}$

(b) \textbf{No}. Extrapolation beyond the data range $(t = 10 > 6)$ is unreliable. Polynomial interpolation can give erratic results outside the data interval.

(c) With 4 points, the polynomial has degree at most $\boxed{3}$ (cubic).
\end{solution}

\begin{solution}
\textbf{Question 4}

(a) Divided differences for $(0,1)$, $(1,0)$, $(2,1)$:

$f[x_0] = 1$

$f[x_0,x_1] = \frac{0-1}{1-0} = -1$

$f[x_1,x_2] = \frac{1-0}{2-1} = 1$

$f[x_0,x_1,x_2] = \frac{1-(-1)}{2-0} = 1$

$P_2(x) = 1 + (-1)(x-0) + 1(x-0)(x-1) = 1 - x + x^2 - x = \boxed{x^2 - 2x + 1}$

(b) Adding $(3, 4)$:

$f[x_2,x_3] = \frac{4-1}{3-2} = 3$

$f[x_1,x_2,x_3] = \frac{3-1}{3-1} = 1$

$f[x_0,x_1,x_2,x_3] = \frac{1-1}{3-0} = 0$

$P_3(x) = P_2(x) + 0 \cdot (x)(x-1)(x-2) = \boxed{x^2 - 2x + 1}$

\textbf{Advantage}: Newton's form allows adding new points without recalculating the entire polynomial---just compute new divided differences and add a term.
\end{solution}

\begin{solution}
\textbf{Question 5}

(a) Trapezoidal with $n = 2$: $h = \frac{\pi}{2}$

$I \approx \frac{h}{2}[f(0) + 2f(\pi/2) + f(\pi)] = \frac{\pi/2}{2}[0 + 2(1) + 0] = \frac{\pi}{2} \approx \boxed{1.5708}$

(b) Simpson's with $n = 2$: $h = \frac{\pi}{2}$

$I \approx \frac{h}{3}[f(0) + 4f(\pi/2) + f(\pi)] = \frac{\pi/2}{3}[0 + 4(1) + 0] = \frac{2\pi}{3} \approx \boxed{2.0944}$

(c) Exact: $\int_0^\pi \sin(x)\, dx = [-\cos(x)]_0^\pi = -(-1) - (-1) = 2$

Trapezoidal error: $|1.5708 - 2| = 0.429$ (21\% error)

Simpson's error: $|2.0944 - 2| = 0.094$ (4.7\% error)

\textbf{Simpson's is more accurate} due to higher order ($O(h^4)$ vs $O(h^2)$).
\end{solution}

\begin{solution}
\textbf{Question 6}

(a) Linear interpolant through $(a, f(a))$ and $(b, f(b))$:

$P_1(x) = f(a) + \frac{f(b)-f(a)}{b-a}(x-a)$

$\int_a^b P_1(x)\, dx = \int_a^b \left[f(a) + \frac{f(b)-f(a)}{b-a}(x-a)\right] dx$

$= f(a)(b-a) + \frac{f(b)-f(a)}{b-a} \cdot \frac{(b-a)^2}{2}$

$= f(a)(b-a) + \frac{f(b)-f(a)}{2}(b-a)$

$= \frac{b-a}{2}[2f(a) + f(b) - f(a)] = \boxed{\frac{b-a}{2}[f(a) + f(b)]}$

(b) Simpson's rule is exact for polynomials up to degree $\boxed{3}$.

(c) \textbf{Simple}: One trapezoid over $[a,b]$.

\textbf{Composite}: Divide $[a,b]$ into $n$ subintervals, apply trapezoidal to each, sum results. More accurate for same interval.
\end{solution}

\begin{solution}
\textbf{Question 7}

(a) $A = \begin{pmatrix} 2 & 1 \\ 4 & 5 \end{pmatrix}$

$m_{21} = \frac{4}{2} = 2$

$U = \begin{pmatrix} 2 & 1 \\ 0 & 5-2(1) \end{pmatrix} = \begin{pmatrix} 2 & 1 \\ 0 & 3 \end{pmatrix}$

$L = \begin{pmatrix} 1 & 0 \\ 2 & 1 \end{pmatrix}$

$\boxed{L = \begin{pmatrix} 1 & 0 \\ 2 & 1 \end{pmatrix}, \quad U = \begin{pmatrix} 2 & 1 \\ 0 & 3 \end{pmatrix}}$

(b) Solve $L\mathbf{y} = \mathbf{b}$: $\begin{pmatrix} 1 & 0 \\ 2 & 1 \end{pmatrix}\begin{pmatrix} y_1 \\ y_2 \end{pmatrix} = \begin{pmatrix} 5 \\ 13 \end{pmatrix}$

$y_1 = 5$, $2(5) + y_2 = 13 \Rightarrow y_2 = 3$

Solve $U\mathbf{x} = \mathbf{y}$: $\begin{pmatrix} 2 & 1 \\ 0 & 3 \end{pmatrix}\begin{pmatrix} x_1 \\ x_2 \end{pmatrix} = \begin{pmatrix} 5 \\ 3 \end{pmatrix}$

$3x_2 = 3 \Rightarrow x_2 = 1$

$2x_1 + 1 = 5 \Rightarrow x_1 = 2$

$\boxed{x_1 = 2, \quad x_2 = 1}$
\end{solution}

\begin{solution}
\textbf{Question 8}

$f(t,y) = t + y$, $t_0 = 0$, $y_0 = 1$, $h = 0.2$

(a) RK2 (Midpoint):

$k_1 = hf(0, 1) = 0.2(0 + 1) = 0.2$

$k_2 = hf(0.1, 1.1) = 0.2(0.1 + 1.1) = 0.2(1.2) = 0.24$

$y_1 = 1 + k_2 = \boxed{1.24}$

(b) RK4:

$k_1 = 0.2 f(0, 1) = 0.2(1) = 0.2$

$k_2 = 0.2 f(0.1, 1.1) = 0.2(1.2) = 0.24$

$k_3 = 0.2 f(0.1, 1.12) = 0.2(1.22) = 0.244$

$k_4 = 0.2 f(0.2, 1.244) = 0.2(1.444) = 0.2888$

$y_1 = 1 + \frac{1}{6}(0.2 + 2(0.24) + 2(0.244) + 0.2888)$

$= 1 + \frac{1}{6}(1.4568) = 1 + 0.2428 = \boxed{1.2428}$
\end{solution}

\end{document}
