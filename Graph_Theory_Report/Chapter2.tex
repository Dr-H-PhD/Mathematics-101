\chapter{Fundamentals of Graph Theory}
\label{chap:fundamentals}

\section{Basic Definitions}
\subsection{Graphs, Vertices, Edges}
\begin{definition}[Graph]
	A \textbf{graph} $G$ is an ordered pair $(V, E)$ where:
	\begin{itemize}
		\item $V$ is a set of \textbf{vertices} (or nodes)
		\item $E \subseteq \{\{u, v\} \mid u, v \in V, u \neq v\}$ is a set of \textbf{edges} (or links)
	\end{itemize}
	For a graph with $n$ vertices and $m$ edges, we denote $|V| = n$ and $|E| = m$.
\end{definition}

\begin{definition}[Directed Graph]
	A \textbf{directed graph} (or digraph) $D = (V, A)$ consists of:
	\begin{itemize}
		\item $V$: set of vertices
		\item $A \subseteq \{(u, v) \mid u, v \in V, u \neq v\}$: set of \textbf{arcs} (directed edges)
	\end{itemize}
	Each arc $(u, v)$ has a direction from $u$ (tail) to $v$ (head).
\end{definition}

\subsection{Degree, Paths, Cycles}
\begin{definition}[Degree]
	The \textbf{degree} of a vertex $v$, denoted $\deg(v)$, is the number of edges incident to $v$. For directed graphs:
	\begin{itemize}
		\item $\text{indeg}(v)$: number of arcs entering $v$
		\item $\text{outdeg}(v)$: number of arcs leaving $v$
	\end{itemize}
\end{definition}

\begin{definition}[Path]
	A \textbf{path} is a sequence of vertices $v_0, v_1, \dots, v_k$ such that $(v_i, v_{i+1}) \in E$ for $i = 0, \dots, k-1$. The \textbf{length} of the path is $k$ (number of edges).
\end{definition}

\begin{definition}[Cycle]
	A \textbf{cycle} is a path $v_0, v_1, \dots, v_k$ where $v_0 = v_k$ and all other vertices are distinct.
\end{definition}

\subsection{Connectedness}
\begin{definition}[Connected Graph]
	An undirected graph is \textbf{connected} if there exists a path between every pair of vertices.
\end{definition}

\begin{definition}[Connected Components]
	A \textbf{connected component} of a graph is a maximal connected subgraph.
\end{definition}

\section{Types of Graphs}
\subsection{Simple Graphs}
\begin{definition}[Simple Graph]
	A \textbf{simple graph} is an undirected graph without loops (edges connecting a vertex to itself) and without multiple edges between the same pair of vertices.
\end{definition}

\subsection{Multigraphs}
\begin{definition}[Multigraph]
	A \textbf{multigraph} allows multiple edges between the same pair of vertices.
\end{definition}

\subsection{Bipartite Graphs}
\begin{definition}[Bipartite Graph]
	A graph $G = (V, E)$ is \textbf{bipartite} if $V$ can be partitioned into two disjoint sets $U$ and $W$ such that every edge connects a vertex in $U$ to a vertex in $W$.
\end{definition}

\begin{theorem}[Characterization of Bipartite Graphs]
	A graph is bipartite if and only if it contains no odd cycles.
\end{theorem}

\subsection{Complete Graphs}
\begin{definition}[Complete Graph]
	A \textbf{complete graph} $K_n$ is a simple graph with $n$ vertices where every pair of distinct vertices is connected by an edge.
\end{definition}

\subsection{Trees and Forests}
\begin{definition}[Tree]
	A \textbf{tree} is a connected graph with no cycles.
\end{definition}

\begin{theorem}[Tree Properties]
	For a tree $T = (V, E)$ with $n$ vertices:
	\begin{enumerate}
		\item $|E| = n - 1$
		\item There is exactly one path between any two vertices
		\item Adding any edge creates exactly one cycle
		\item Removing any edge disconnects the graph
	\end{enumerate}
\end{theorem}

\subsection{Planar Graphs}
\begin{definition}[Planar Graph]
	A graph is \textbf{planar} if it can be drawn in the plane without edge crossings.
\end{definition}

\begin{theorem}[Euler's Formula]
	For a connected planar graph with $n$ vertices, $m$ edges, and $f$ faces:
	\[ n - m + f = 2 \]
\end{theorem}

\section{Graph Representations}
\subsection{Adjacency Matrix}
The \textbf{adjacency matrix} $A$ of a graph $G = (V, E)$ with $n$ vertices is an $n \times n$ matrix where:
\[ A_{ij} = \begin{cases}
	1 & \text{if } (i, j) \in E \\
	0 & \text{otherwise}
\end{cases} \]

For weighted graphs, $A_{ij} = w(i, j)$ where $w(i, j)$ is the weight of edge $(i, j)$.

\begin{example}
	For the graph with vertices $\{1,2,3\}$ and edges $\{(1,2), (2,3), (3,1)\}$:
	\[ A = \begin{pmatrix}
		0 & 1 & 1 \\
		1 & 0 & 1 \\
		1 & 1 & 0
	\end{pmatrix} \]
\end{example}

\subsection{Adjacency List}
An \textbf{adjacency list} stores for each vertex a list of its neighbors:
\begin{itemize}
	\item Space complexity: $O(n + m)$
	\item Efficient for sparse graphs
	\item Easy to iterate over neighbors
\end{itemize}

\subsection{Edge List}
An \textbf{edge list} simply stores all edges as pairs of vertices:
\begin{itemize}
	\item Space complexity: $O(m)$
	\item Simple implementation
	\item Inefficient for neighbor queries
\end{itemize}

\subsection{Incidence Matrix}
The \textbf{incidence matrix} $B$ of a graph $G = (V, E)$ is an $n \times m$ matrix where:
\[ B_{ij} = \begin{cases}
	1 & \text{if vertex } i \text{ is incident to edge } j \\
	0 & \text{otherwise}
\end{cases} \]

\section{Weighted Graphs}
\begin{definition}[Weighted Graph]
	A \textbf{weighted graph} $G = (V, E, w)$ assigns a weight $w(e) \in \mathbb{R}$ to each edge $e \in E$.
\end{definition}

Applications include:
\begin{itemize}
	\item Transportation networks (distances/travel times)
	\item Communication networks (bandwidth/cost)
	\item Social networks (strength of relationships)
\end{itemize}

\section{Connectivity and Components}
\subsection{Vertex Connectivity}
\begin{definition}[k-Connected]
	A graph is \textbf{k-connected} if removing fewer than $k$ vertices does not disconnect it.
\end{definition}

\subsection{Edge Connectivity}
\begin{definition}[k-Edge-Connected]
	A graph is \textbf{k-edge-connected} if removing fewer than $k$ edges does not disconnect it.
\end{definition}

\section{Special Graph Families}
\subsection{Regular Graphs}
\begin{definition}[Regular Graph]
	A graph is \textbf{k-regular} if every vertex has degree $k$.
\end{definition}

\subsection{Cayley Graphs}
\begin{definition}[Cayley Graph]
	Given a group $G$ and a generating set $S$, the \textbf{Cayley graph} has vertices corresponding to group elements and edges $(g, gs)$ for $g \in G, s \in S$.
\end{definition}

\section{Planar Graphs}
\subsection{Kuratowski's Theorem}
\begin{theorem}[Kuratowski, 1930]
	A graph is planar if and only if it does not contain a subdivision of $K_5$ or $K_{3,3}$.
\end{theorem}

\subsection{Four Color Theorem}
\begin{theorem}[Four Color Theorem]
	Every planar graph can be colored with at most 4 colors such that no adjacent vertices share the same color.
\end{theorem}

\section{Cliques}
\begin{definition}[Clique]
	A \textbf{clique} is a subset of vertices such that every two distinct vertices are adjacent.
\end{definition}

\begin{definition}[Clique Number]
	The \textbf{clique number} $\omega(G)$ is the size of the largest clique in $G$.
\end{definition}

\section{Independent Sets}
\begin{definition}[Independent Set]
	An \textbf{independent set} is a subset of vertices such that no two vertices are adjacent.
\end{definition}

\begin{definition}[Independence Number]
	The \textbf{independence number} $\alpha(G)$ is the size of the largest independent set in $G$.
\end{definition}

\begin{theorem}[Relationship]
	For any graph $G$ with $n$ vertices:
	\[ \omega(G) \cdot \alpha(G) \geq n \]
\end{theorem}