\documentclass[12pt,a4paper]{report}
\usepackage[utf8]{inputenc}
\usepackage[T1]{fontenc}
\usepackage{amsmath, amssymb, amsthm}
\usepackage{graphicx}
\usepackage{hyperref}
\usepackage{algorithm}
\usepackage{algorithmic}
\usepackage{caption}
\usepackage{subcaption}
\usepackage{booktabs}
\usepackage{multirow}
\usepackage{array}
\usepackage{geometry}
\usepackage{enumitem}
\usepackage{titlesec}
\usepackage{color}
\usepackage{xcolor}
\usepackage{listings}
\usepackage{float}
\usepackage{algorithm}
\usepackage{algorithmic}
\usepackage{tikz}
\usetikzlibrary{positioning, shapes, arrows, calc}
\usepackage{algpseudocode}  % For \Procedure, \EndProcedure
\usepackage{algorithm}      % For algorithm environment

% ========== PAGE LAYOUT ==========
\geometry{left=2.5cm, right=2.5cm, top=2.5cm, bottom=2.5cm}

% ========== HYPERREF SETUP ==========
\hypersetup{
	colorlinks=true,
	linkcolor=blue,
	filecolor=magenta,      
	urlcolor=cyan,
	citecolor=green,
	pdftitle={Mathematics for Data Science: Graph Theory and Applications},
	pdfauthor={Student Name},
	pdfsubject={Mathematics, Data Science, Graph Theory},
	pdfkeywords={Graph Theory, Machine Learning, Data Science, Graph Neural Networks}
}

% ========== CODE LISTING COLORS ==========
\definecolor{codegreen}{rgb}{0,0.6,0}
\definecolor{codegray}{rgb}{0.5,0.5,0.5}
\definecolor{codepurple}{rgb}{0.58,0,0.82}
\definecolor{backcolour}{rgb}{0.95,0.95,0.92}

% ========== CODE LISTING STYLE ==========
\lstdefinestyle{mystyle}{
	backgroundcolor=\color{backcolour},   
	commentstyle=\color{codegreen},
	keywordstyle=\color{magenta},
	numberstyle=\tiny\color{codegray},
	stringstyle=\color{codepurple},
	basicstyle=\ttfamily\footnotesize,
	breakatwhitespace=false,         
	breaklines=true,                 
	captionpos=b,                    
	keepspaces=true,                 
	numbers=left,                    
	numbersep=5pt,                  
	showspaces=false,                
	showstringspaces=false,
	showtabs=false,                  
	tabsize=2
}
\lstset{style=mystyle}

% ========== THEOREM ENVIRONMENTS ==========
\theoremstyle{definition}
\newtheorem{definition}{Definition}[chapter]
\newtheorem{theorem}{Theorem}[chapter]
\newtheorem{lemma}[theorem]{Lemma}
\newtheorem{corollary}[theorem]{Corollary}
\newtheorem{example}{Example}[chapter]
\newtheorem{remark}{Remark}[chapter]

% ========== CHAPTER/SECTION FORMATTING ==========
\titleformat{\chapter}[display]
{\normalfont\huge\bfseries}
{\chaptertitlename\ \thechapter}{20pt}{\Huge}
\titlespacing*{\chapter}{0pt}{-30pt}{40pt}

\titleformat{\section}
{\normalfont\Large\bfseries}
{\thesection}{1em}{}

\titleformat{\subsection}
{\normalfont\large\bfseries}
{\thesubsection}{1em}{}

\titleformat{\subsubsection}
{\normalfont\normalsize\bfseries}
{\thesubsubsection}{1em}{}

% ========== LINE SPACING ==========
\renewcommand{\baselinestretch}{1.2}

% ========== CUSTOM COMMANDS ==========
\newcommand{\code}[1]{\texttt{#1}}
\newcommand{\mat}[1]{\mathbf{#1}}
\newcommand{\set}[1]{\mathbb{#1}}
\newcommand{\R}{\mathbb{R}}
\newcommand{\Z}{\mathbb{Z}}
\newcommand{\N}{\mathbb{N}}
\newcommand{\graph}{\mathcal{G}}

\begin{document}
	
	% ========== TITLE PAGE ==========
\begin{titlepage}
	\centering
	
	% ========== LOGO SECTION ==========
	\IfFileExists{/Users/mac/Desktop/Mathematics-101/Graph_Theory_Report/images/ECE_LOGO_2021_web.png}{
		\includegraphics[width=0.3\textwidth]{/Users/mac/Desktop/Mathematics-101/Graph_Theory_Report/images/ECE_LOGO_2021_web.png}\\[2cm]
	}{
		\vspace*{2cm}
	}
	
	% ========== MAIN TITLE ==========
	\rule{\textwidth}{1.5pt}\\[0.4cm]	
\begin{center}
	{\fontsize{36}{40}\selectfont \textbf{GRAPH THEORY}}\\
\end{center}
	\rule{\textwidth}{1.5pt}\\[2cm]
\vspace{1cm}

	
	% ========== AUTHOR & SUPERVISOR ==========
	\begin{minipage}[t]{0.45\textwidth}
		\raggedright
		\large
		\textbf{AUTHORS:}\\[0.2cm]
		\textbf{\textsc{-Anir Agountaf}}\\[0.5cm]
		\textbf{\textsc{-Muhammad Hassaan Shafique}}\\
		
	\end{minipage}
	\hfill
	\vrule
	\hfill
	\begin{minipage}[t]{0.45\textwidth}
		\raggedleft
		\large
		\textbf{PROFESSOR:}\\[0.2cm]
		\textbf{M. \textsc{Soufian Benamor}}\\[0.5cm]
		\normalsize
		\textbf{COURSE:}\\
		\small
		Mathematics for Data Science
		\normalsize
		\textbf{PROGRAM:}\\
		\small
		Master of Science in Data Management
	\end{minipage}
	
	% ========== SPACING ==========
	\vfill
	
	% ========== SUBMISSION INFO ==========
	\begin{center}
		\rule{0.8\textwidth}{0.4pt}\\[0.5cm]
		{\large \textbf{Academic Year 2025-2026}}\\[0.3cm]
		{\large \textbf{Project Report}}\\[0.8cm]
		{\Large \textbf{7 December 2025}}
	\end{center}
	
	\vspace*{1cm}
	
\end{titlepage}
	
	% ========== ABSTRACT ==========
	\chapter*{Abstract}
	\addcontentsline{toc}{chapter}{Abstract}
	Graph theory is a foundational area of discrete mathematics concerned with the study of networks formed by nodes (vertices) and the connections (edges) between them. This report provides an in-depth overview of core graph-theoretic concepts, including graph representations, connectivity, traversals, shortest-path algorithms, tree and cycle structures and properties of directed and weighted graphs. Beyond these fundamentals, the report emphasizes the growing relevance of graph theory in modern computational fields, particularly artificial intelligence (AI).
	
	In AI, graphs serve as powerful modeling tools for capturing complex relationships within data. Knowledge graphs represent structured information that supports reasoning and semantic understanding in intelligent systems. Graph algorithms enable efficient navigation and inference over these structures, contributing to advancements in natural language processing, recommendation systems and search technologies. Additionally, the emergence of graph-based machine learning methods---such as Graph Neural Networks (GNNs)---has expanded the role of graph theory in pattern recognition, social network analysis, molecular prediction and other areas where relational dependencies are critical.
	
	By examining both classical theory and contemporary AI applications, this project highlights how graph theory provides the mathematical framework necessary for representing, analyzing and learning from interconnected data. The report concludes that as AI systems increasingly rely on relational information, the importance of graph theory will continue to grow, bridging the gap between mathematical abstraction and real-world intelligent technologies.
	
	
	\vspace{0.5cm}
	\noindent\textbf{Keywords:} Graph Theory, Data Science, Artificial Intelligence, Graph Neural Networks, Network Analysis, Machine Learning, Algorithms
	
	% ========== TABLE OF CONTENTS ==========
	\tableofcontents
	\listoffigures
	\addcontentsline{toc}{chapter}{List of Figures}
	\listoftables
	\addcontentsline{toc}{chapter}{List of Tables}
	\listofalgorithms
	\addcontentsline{toc}{chapter}{List of Algorithms}
	
	% ========== MAIN CONTENT ==========

	
	% Include all chapters from your original structure
	\chapter{Introduction}

Graph theory is a branch of discrete mathematics concerned with the study of structures composed of \textbf{vertices} (nodes) and \textbf{edges} (connections). Although originating with Euler’s 1736 solution to the Königsberg Bridge Problem \cite{euler1736}, it has evolved into a major mathematical discipline with extensive applications in modern computer science, data science, network analysis and artificial intelligence.

\section{Why Graph Theory Matters in Data Science}

Many datasets today are relational rather than tabular.  
Examples include:
\begin{itemize}
	\item Social networks — users connected via friendships.
	\item Web graphs — websites connected by hyperlinks.
	\item Transportation systems — cities connected by routes.
	\item Biological networks — proteins interacting with one another.
\end{itemize}

Traditional statistical methods often fail to capture this relational structure. Graph theory provides:
\begin{itemize}
	\item Mathematical tools for modeling connected systems.
	\item Algorithms for efficient search, traversal and pattern discovery.
	\item Foundations for emerging fields such as Graph Neural Networks (GNNs) \cite{kipf2017semi}.
\end{itemize}

\section{Basic Graph Concept Illustration}

Figure~\ref{fig:simple_graph} shows a basic undirected graph.

\begin{figure}[H]
	\centering
	\begin{tikzpicture}[node distance=2cm, every node/.style={circle, draw, fill=blue!20}]
		\node (A) {A};
		\node (B) [right of=A] {B};
		\node (C) [below of=A] {C};
		\node (D) [below of=B] {D};
		
		\draw (A)--(B);
		\draw (A)--(C);
		\draw (B)--(C);
		\draw (B)--(D);
		\draw (C)--(D);
	\end{tikzpicture}
	\caption{A sample undirected graph with four vertices.}
	\label{fig:simple_graph}
\end{figure}

\section{Structure of This Report}

This report is divided into five chapters:

\begin{itemize}
	\item \textbf{Chapter 1}: Introduction to graph theory and its relevance.
	\item \textbf{Chapter 2}: Formal mathematical foundations and graph structures.
	\item \textbf{Chapter 3}: Classical graph algorithms used in data science.
	\item \textbf{Chapter 4}: Advanced topics including spectral graph theory and GNNs.
	\item \textbf{Chapter 5}: Conclusions and implications for data-driven applications.
\end{itemize}  % Introduction
	\chapter{Fundamentals of Graph Theory}

This chapter establishes the mathematical foundations needed to work with graph structures in data science. It introduces formal graph definitions, graph types, matrix representations and key structural concepts such as degree, connectivity and cycles.

\section{Basic Definitions}

\begin{definition}
	A \textbf{graph} is an ordered pair $\graph=(V,E)$ where:
	\[
	V=\{v_1,\ldots,v_n\} \quad \text{and} \quad E \subseteq V \times V.
	\]
\end{definition}

Edges may be \textbf{directed}, \textbf{undirected}, \textbf{weighted} or \textbf{unweighted}.  
Applications like social networks typically use undirected graphs, while recommendation systems or web link structures use directed, weighted graphs \cite{newman2010networks}.

\section{Degree and Neighborhood}

For an undirected graph:
\[
\deg(v) = |\{u \in V : (u,v)\in E\}|.
\]

For a directed graph:
\[
\deg^+(v)=\text{out-degree}, \qquad \deg^-(v)=\text{in-degree}.
\]

\section{Graph Representations}

Graphs can be represented in multiple ways.

\subsection{Adjacency Matrix}

\[
A_{ij} = 
\begin{cases}
	1 & \text{if } (v_i,v_j)\in E,\\
	0 & \text{otherwise.}
\end{cases}
\]

\begin{table}[H]
	\centering
	\caption{Adjacency Matrix of a Simple Graph}
	\begin{tabular}{c|cccc}
		& A & B & C & D \\
		\hline
		A & 0 & 1 & 1 & 0 \\
		B & 1 & 0 & 1 & 1 \\
		C & 1 & 1 & 0 & 1 \\
		D & 0 & 1 & 1 & 0 \\
	\end{tabular}
\end{table}

\subsection{Adjacency List}

For sparse graphs, adjacency lists are far more memory-efficient.

\section{Graph Traversal Algorithms}

Traversal algorithms are essential in data science — from exploring social networks to crawling web graphs.

Below are corrected versions of BFS and DFS in proper algorithm floats.

\begin{algorithm}[H]
	\caption{Breadth-First Search (BFS)}
	\begin{algorithmic}[1]
		\Procedure{BFS}{$G, s$}
		\State mark $s$ as visited
		\State enqueue $s$
		\While{queue not empty}
		\State $v \gets$ dequeue()
		\For{each neighbor $u$ of $v$}
		\If{$u$ not visited}
		\State mark $u$
		\State enqueue $u$
		\EndIf
		\EndFor
		\EndWhile
		\EndProcedure
	\end{algorithmic}
\end{algorithm}

\begin{algorithm}[H]
	\caption{Depth-First Search (DFS)}
	\begin{algorithmic}[1]
		\Procedure{DFS}{$G, v$}
		\State mark $v$ as visited
		\For{each neighbor $u$ of $v$}
		\If{$u$ not visited}
		\State DFS($G,u$)
		\EndIf
		\EndFor
		\EndProcedure
	\end{algorithmic}
\end{algorithm}

\section{Connectivity and Components}

A graph is \textbf{connected} if every pair of vertices is linked by some path.  
Connected components can be identified efficiently with BFS or DFS.

\[
\text{Number of connected components} = k
\]

\section{Cycles and Trees}

A \textbf{tree} is a connected, acyclic graph.  
Fundamental property:
\[
\text{A tree with } n \text{ vertices has } n-1 \text{ edges.}
\]

Trees are widely used in machine learning (e.g., decision trees, hierarchical clustering).  % Fundamentals of Graph Theory
	 \chapter{Network Science and Complex Networks}
\label{ch:networkscience}

Network science extends classical graph theory to the study of large, real-world networks.  
Such networks often exhibit properties like clustering, heavy-tailed degree distributions, and small-world behavior \cite{newman2010networks}.

% ---------------------------------------------------------
\section{Random Graph Models}

\subsection{Erdős–Rényi Model}

In the $G(n,p)$ model, each pair of vertices is connected with probability $p$.  
This simple model facilitates theoretical analysis but rarely matches real data.

\subsection{Watts–Strogatz Model}

Designed to reproduce the ``small-world'' phenomenon:  
high clustering and short path lengths are typical of social systems.

\subsection{Barabási–Albert Model}

Models scale-free networks where degree distributions follow a power law:
\[
P(k) \propto k^{-3}.
\]

Such networks arise in the web graph, biological systems, and city transportation \cite{barabasi2016network}.

% ---------------------------------------------------------
\section{Centrality Measures}

Centrality quantifies importance of nodes in a network.

\subsection{Degree Centrality}

Measures local influence:
\[
C_D(v) = \deg(v).
\]

\subsection{Closeness Centrality}

Measures how close a node is to others:
\[
C_C(v) = \frac{1}{\sum_{u} d(v,u)}.
\]

\subsection{Betweenness Centrality}

Measures how often a node lies on shortest paths:
\[
C_B(v) = \sum_{s,t} \frac{\sigma_{st}(v)}{\sigma_{st}}.
\]

\subsection{Eigenvector Centrality and PageRank}

Eigenvector centrality measures influence recursively:  
high-scoring nodes connect to high-scoring neighbors.  
PageRank modifies this with damping for web ranking \cite{brin1998anatomy}.

% ---------------------------------------------------------
\section{Community Detection}

Communities are densely connected node groups.

\subsection{Modularity}

Measures how well a partition captures community structure:
\[
Q = \frac{1}{2m} \sum_{ij} (A_{ij} - \frac{k_i k_j}{2m})\delta(c_i,c_j).
\]

Algorithms such as Louvain and Leiden optimize modularity.

% ---------------------------------------------------------
\section{Summary}

This chapter introduced models and measures that help analyze real network behavior.  
Network science provides tools essential for identifying influential nodes, community structure, and overall graph organization.  % Classical Graph Algorithms
	 \chapter{Advanced Graph Theory Concepts}

This chapter covers modern graph concepts highly relevant to machine learning and data science.

\section{Spectral Graph Theory}

Spectral graph theory studies the eigenvalues and eigenvectors of matrices such as the Laplacian:
\[
L = D - A.
\]

The second-smallest eigenvalue $\lambda_2$, known as the \textbf{Fiedler value}, measures graph connectivity.

\section{Random Walks on Graphs}

Random walks are used in:
\begin{itemize}
	\item PageRank \cite{brin1998pagerank}
	\item Node2Vec embeddings
	\item Semi-supervised learning on graphs
\end{itemize}

\[
P_{ij} = \frac{A_{ij}}{\deg(i)}.
\]

\section{Graph Neural Networks (GNNs)}

GNNs generalize convolution to irregular graph domains.  
A typical GCN layer \cite{kipf2017semi} is:

\[
H^{(l+1)} = \sigma\left( \tilde{D}^{-1/2}\tilde{A}\tilde{D}^{-1/2} H^{(l)} W^{(l)} \right).
\]

Applications:
\begin{itemize}
	\item Molecule property prediction
	\item Fraud detection
	\item Recommender systems
\end{itemize}

\section{Knowledge Graphs}

Knowledge graphs represent semantic relationships using triples:
\[
(h, r, t)
\]
where $h$ is the head entity, $t$ is the tail and $r$ is the relation.  
They power modern AI systems including Google Knowledge Graph and large language models.  % Advanced Graph Theory Concepts
	 \chapter{Graph Theory in Machine Learning}

\section{Graph-Based Learning Paradigms}
\subsection{Transductive Learning}
\subsection{Semi-Supervised Learning}
\subsection{Graph Regularization}

\section{Feature Engineering with Graphs}
\subsection{Graph Embeddings}
\subsection{Node2Vec}
\subsection{DeepWalk}
\subsection{Graph Kernels}

\section{Graph Neural Networks (GNNs)}
\subsection{Message Passing Neural Networks}
\subsection{Graph Convolutional Networks (GCN)}
\subsection{Graph Attention Networks (GAT)}
\subsection{GraphSAGE}
\subsection{Training Challenges and Limitations}

\section{Explainability in GNNs}
\subsection{Subgraph Explanations}
\subsection{Feature Attribution}
\subsection{Global vs Local Explainability}  % Graph Theory in Machine Learning
	 \chapter{Applications of Graph Theory in Data Science}

\section{Social Network Analysis}
\subsection{Community Detection}
\subsection{Influence Propagation}
\subsection{Centrality for Influence Ranking}

\section{Recommendation Systems}
\subsection{Graph-Based Collaborative Filtering}
\subsection{Heterogeneous Information Networks}
\subsection{Knowledge Graphs}

\section{Natural Language Processing}
\subsection{Text as Graphs}
\subsection{Dependency Parsing Graphs}
\subsection{Graph Embeddings for NLP}

\section{Computer Vision}
\subsection{Scene Graphs}
\subsection{Graph-Based Image Segmentation}

\section{Fraud Detection}
\subsection{Transaction Networks}
\subsection{Anomaly Detection on Graphs}

\section{Biological and Chemical Graphs}
\subsection{Protein Interaction Networks}
\subsection{Molecular Graphs}
\subsection{Drug Discovery with GNNs}  % Applications of Graph Theory in Data Science
	 \chapter{Graph Databases and Large-Scale Graph Processing}

\section{Graph Database Models}
\subsection{Property Graph Model}
\subsection{RDF Model}

\section{Graph Query Languages}
\subsection{Cypher}
\subsection{SPARQL}
\subsection{Gremlin}

\section{Distributed Graph Systems}
\subsection{Apache Spark GraphX}
\subsection{Google Pregel Model}
\subsection{Graph Processing Frameworks}  % Graph Databases and Large-Scale Graph Processing
	 \chapter{Case Studies}

\section{Social Media User Classification}
\section{Traffic Prediction Using GNNs}
\section{Fraud Detection in Financial Networks}
\section{Drug Discovery Using Molecular Graph Learning}  % Case Studies
	 \chapter{Conclusion}

\section{Summary of Findings}
\section{Future Directions in Graph-Based AI}
\section{Limitations of Graph Methods}  % Conclusion
	
	% ========== BIBLIOGRAPHY ==========
	\bibliographystyle{ieeetr}
	\bibliography{references}
	\addcontentsline{toc}{chapter}{Bibliography}
	
\end{document}